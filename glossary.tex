\newacronym{coh}{COH}{Village of Cornwall-on-Hudson}
%\newacronym{bg}{BG}{Town of Blooming Grove}
%\newacronym{corn}{Cornwall}{Town of Cornwall}
\newacronym{ocwa}{OCWA}{Orange County Water Authority}
\newacronym{hrep}{HREP}{Hudson River Estuary Program}
\newacronym{nynhp}{NYNHP}{New York Natural Heritage Program}
\newacronym{rt94}{RT 94}{NYS Route 94}
\newacronym{rt52}{CO RT 52}{County Route 52}
\newacronym{amt}{AMT}{Above Mean Terrain}
%\newacronym{vwash}{Village of Washingtonville}{Village of Washingtonville}
\newacronym{bap}{BAP}{Biological Assessment Profile}
\newacronym{naaqs}{NAAQS}{National Ambient Air Quality Standard}
\newacronym{epa}{EPA}{Environmental Protection Agency}
\newacronym{csc}{CSC}{Climate Smart Communities}
\newacronym{nri}{NRI}{Natural Resources Inventory}
\newacronym{gis}{GIS}{Geographic Information System}
\newacronym{cac}{CAC}{Conservation Advisory Council}
\newacronym{iba}{IBA}{Important Bird Area}
\newacronym{sba}{SBA}{Significant Biodiversity Areas} 
\newacronym{noaa}{NOAA}{National Oceanic and Atmospheric Administration}
\newacronym{esa}{ESA}{Endangered Species Act}
\newacronym{fema}{FEMA}{Federal Emergency Management Agency}
\newacronym{hud}{HUD}{Department of Housing and Urban Development}
\newacronym{spdes}{SPDES}{State Permit Discharge Elimination System}
\newacronym{ribs}{RIBS}{Rotating Integrated Basic Studies}
\newacronym{nysta}{NYSTA}{New York State Thruway Authority}
\newacronym{nysdec}{NYSDEC}{New York State Department of Environmental Conservation}
\newacronym{nwi}{NWI}{National Wetlands Inventory}

%%%% from paper %%%
%%% The glossary entry the acronym links to   
%\newacronym{ghg}{GHG}{Green House Gas}

\newglossaryentry{savg}{name={SAV},
    description={Plants that are always under water. The most common native species of SAV in the Hudson River watershed is water celery (\textit{Vallisneria americana}), but other species include clasping leaved pondweed (\textit{Potamogeton perfoliatus}), and such non-native plants as curly pondweed (\textit{Potamogeton crispus}) and Eurasian water milfoil (\textit{Myriophyllum spicatum})}}

%%% define the acronym and use the see= option
\newglossaryentry{sav}{type=\acronymtype, name={SAV}, description={Submerged Aquatic Vegetation}, first={Submerged Aquatic Vegetation (SAV)\glsadd{savg}}, see=[Glossary:]{savg}}

\newglossaryentry{ghgg}{name={GHG},
    description={Gases which absorb and emit infrared radiation in the wavelength range that is emitted by Earth. Some of the most common green house gases on earth include: C0$_{2}$, N$_{}$O, CH$_{4}$, CFCs, O$_{3}$ and HFCs, HCFCs}}

\newglossaryentry{ghg}{type=\acronymtype, name={GHG}, description={Green House Gas}, first={green house gas (GHG)\glsadd{ghgg}}, see=[Glossary:]{ghgg}}

%%%% glossary %%%%%
\newglossaryentry{calcareous soils}{
    name=calcareous soils,
    description={A by-product of Calcareous grassland is a soil containing much calcium carbonate from underlying chalk or limestone rock. Calcareous, or alkaline, soils are often associated with uncommon habitats and rare species. Given their relatively high pH (7.6 to 8.4) they are not ideal for most agricultural uses}
}

\newglossaryentry{glacial outwash soils}{
    name=glacial outwash soils,
    description={An outwash plain is a plain formed of glacial sediments, deposited by meltwater outwash at the edge of a glacier. This soil type is dominated by silt, rock outcrops, gravel, and gravelly silt}
}

\newglossaryentry{green house gases}{
    name=green house gases,
    description={Gases which absorb and emit infrared radiation in the wavelength range that is emitted by Earth. Some of the most common green house gases on earth include: C0$_{2}$, N$_{}$O, CH$_{4}$, CFCs, O$_{3}$ and HFCs, HCFCs}
}
\newglossaryentry{significant natural communities}{
    name=significant natural communities,
    description={Significant Natural Communities are defined by the NYSDEC New York Natural Heritage Program as locations of rare or high-quality wetlands, forests, grasslands, ponds, streams, and other types of habitats, ecosystems, and ecological areas}
}

\newglossaryentry{talus}{
    name=talus,
    description={a slope formed especially by an accumulation of rock debris, or rock debris at the base of a cliff}
}

\newglossaryentry{habitat}{
    name=habitat,
    description={
    the place or environment where a plant or animal naturally or normally lives and grows}
}

\newglossaryentry{biotope}{
    name=biotope,
    description={a region uniform in environmental conditions and in its populations of animals and plants for which it is the habitat}
}

\newglossaryentry{watershed}{
    name=watershed,
    description={a region or area bounded peripherally by a divide and draining ultimately to a particular watercourse or body of water}
}

\newglossaryentry{tidal wetlands}{
    name=tidal wetlands,
    description={Tidal wetlands are areas consistently covered by water during at least some tide stages. There are many different categories of tidal wetlands depending on the type of vegetation present and the amount of water during high and low tides. New York State uses \href{http://www.dec.ny.gov/lands/5120.html}{specific categories and codes} to describe and represent different types of coastal, tidal and freshwater wetlands}
}


\newglossaryentry{grasslands}{
    name=grasslands,
    description={areas where the vegetation is dominated by grasses (\textit{Poaceae}); however, sedge (\textit{Cyperaceae}) and rush (\textit{Juncaceae}) families can also be found along with variable proportions of legumes, like clover, and other herbs. Grasslands tend to be native habitats created by fire, flooding, or the presence of shallow rocky soil incapable of supporting significant forest growth}
}


\newglossaryentry{meadow}{
    name=meadow,
    description={a field habitat vegetated by grass and other non-woody plants. Meadows may be naturally occurring or artificially created from cleared shrub or woodland. Agricultural land that is not grazed and allowed to grow seasonally for the production of hay is often referred to as meadow}
}

\newglossaryentry{shrubland}{
    name=shrubland,
    description={(scrubland, scrub, brush, or bush) is a plant community characterized by vegetation dominated by shrubs, often also including grasses, herbs, and geophytes. Shrubland may either occur naturally or be the result of human activity}
}



%\newglossaryentry{climate}{
%    name=climate,
%    description={}
%}
%\newglossaryentry{weather}{
%    name=weather,
%    description={}
%}

%\newglossaryentry{extreme temperature}{
%    name=extreme temperature,
%    description={}
%}
%
%\newglossaryentry{aquifer}{
%    name=aquifer,
%    description={}
%}