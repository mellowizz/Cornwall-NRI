\paragraph{Why inventory our resources?} Residents of the Town of Cornwall 
and Village of Cornwall-on-Hudson love and appreciate the scenic beauty and 
calm of their community. We are never far from a mountain to walk or hike, a 
waterbody to enjoy, a tree to seek shade under, or a scenic road to bike on our 
way to a local restaurant or an historic site.
\par
Our natural resources are part and parcel of our quality of life. As such, our 
natural surroundings are worth preserving, and not just for scenic purposes. 
Natural resources are also instrumental to supporting our tourism economy, 
providing clean and plentiful drinking water and clean air, moderating 
temperature, filtering pollutants, absorbing floodwaters, and providing habitat 
for pollinators. By learning about the location and condition of these 
resources, we can plan for our community’s future and ensure we continue to 
receive these nature-based benefits.
\par
The Cornwall \gls{nri} provides a baseline of information on our natural and 
cultural resources. It is meant to help municipal officials, developers, and 
residents make informed land use decisions that have the least negative impact 
on our resources as well as identify areas where our municipalities can apply 
improved conservation measures. The Cornwall NRI can also be used as a 
learning tool by the general public and the Cornwall Central School District as 
it touches on many subjects.
\paragraph{What is a natural resources inventory?} A NRI is a compilation of 
maps and descriptions of important, naturally-occurring resources and cultural 
resources in a given area New York State supports a municipality’s interest in 
understanding and safeguarding its natural and cultural resources under 
\href{https://www.nysenate.gov/legislation/laws/GMU/239-X}{General Municipal Law 
Article 12-F Sections 239-X/-Y}. NYS Town and Village laws also permit the 
incorporation of an NRI into a municipality’s comprehensive plan as a means of 
formalizing the documentation of resources and informing a municipality’s 
planning and zoning 
(\href{https://www.nysenate.gov/legislation/laws/TWN/272-A}{Town Law Section 
272-A} and \href{https://www.nysenate.gov/legislation/laws/VIL/7-722}{Village 
Law Section 7-722}). The Cornwall NRI allows us to visualize our land use 
patterns through maps of habitats and wildlife; aquifers, wetlands, and stream 
health; geology and soils; climate conditions and projections; and historic and 
cultural features. Also included are maps related to the fossil fuel industry, 
which can impact our community. Federal, state, and county agencies provided the 
digital data that was used to develop each map using \gls{gis}; data sources 
appear on each map. As NRIs are not meant to be static documents, the Cornwall 
\gls{cac} commits to updating the maps within five years to ensure data are 
current and new data sets are considered. Additionally, the Cornwall CAC is 
interested in identifying and mapping additional wetlands through a citizen 
science project. Working with the School District on such a project would be 
exciting and beneficial to students and our community.
\paragraph{How can a natural resources inventory be used?} The NRI maps and 
report help us understand how our land use decisions relate to each other and 
impact natural resources across our Town and Village. The Cornwall NRI can 
be used for general planning purposes by municipal planning officials and 
consultants, developers, and residents; all are encouraged to reference the 
maps as a preliminary measure for any proposed development to identify site 
features and constraints. (Please note, however, that the maps are not intended 
to replace site visits or survey requirements as identified by zoning codes.) 
Furthermore, the Cornwall NRI can serve as the foundation for the 
development of measures to better protect our quality of life and natural 
resources through various regulatory and non-regulatory tools, such as laws, 
overlay districts, supplemental zoning standards, and open space planning. The 
NRI also provides a “big-picture” view of natural systems, such as how 
streams flow across the municipalities or how large forests span our border 
with our neighboring Town of Blooming Grove. The development of an NRI, and 
its incorporation into a municipality's comprehensive plan, also clearly 
signals a community's interest in preventing the unintended loss of its natural 
assets to potential public and private funders.
\paragraph{Where can I view the Cornwall NRI?} The Cornwall NRI will be 
available on the Town’s and Village’s website. Hardcopies also will be 
available at the Cornwall Public Library and at Town and Village Halls. 
Comments can be left at either the Town or Village Hall, to the attention of 
the Cornwall CAC, or sent to 
\href{mailto:cornwallnycac@gmail.com}{cornwallnycac@gmail.com}.